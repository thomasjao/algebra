\chapter{排列與組合}
\section{基本定理}
\section{排列之定義 (Permuation)}
\section{排列之公式}
\section{循環排列 (Circular Permutaiton)}
\section{重複排列}
\section{物件不盡相異之排列}
\section{組合之定義 (Combination)}
\section{組合之公式}
\section{求 $_nC_r$ 之最大值}
\section{組合總數}
\section{重複組合}
由 $n$ 種不同物件中,每次取 $r$ 個為一組,若各組中,每種物任可重複取用 2 次、3 次, $\cdots$ 及至 $r$ 次則此種組合,稱為\textbf{重複組合},重複組合中,每種物件因可重複取用,故 $r$ 次數值,可大於 $n$。

\textbf{定理一}:由 $n$ 個不同之物件中,每次取 $r$ 個為一組,如每種物件可重複取 $r$ 次,其重複組合數,常以 $_nH_r$ 表示之,且
\begin{equation*}
_nH_r=_{n+r-1}C_r
\end{equation*}
(證) 如有 $a_1,a_2,a_3$ 三個物件,每次取二個之重複組合,吾人可將每種組合中之物件次序,按下標之大小排定為 $a_1a_1,a_1a_2,a_1a_3,a_2a_2,a_2a_3,a_3a_3$;共有 6 種,即 $_3H_2=6$

然如將各下標依次加 0, 1 則得 $a_1a_2,a_1a_3,a_1a_4,a_2a_3,a_2a_4,a_3a_4$,即變成由 $a_1,a_2,a_3,a_4$ 四個物件中,每次取二個之普通組合。因其組合數亦為 6 種,故得 $_3H_2=_cC_$,亦即 $_3H_2=_{3+2-1}C_2$。

換言之,設有 $a_1,a_2,a_3,\cdots,a_n n$ 個物件,每次取 $r$ 個之重複組合中,取出一種而如前法排定為 $a_1a_1a_2a_2a_2\cdots a_na_n$。再將下標依次加上 0, 1, 2,$\cdots r-1$ 即得 $a_1a_2a_4a_5a_6\cdots a_{n+r-2},a_{n+r-1}$,其中各物件變為不重複,且下標之最大數字由 $n$ 變為 $n+r-1$,故
\begin{equation*}
_nH_r=_{n+r-1}C_r=\dfrac{n(n+1)(n+3)\cdots(n+r-1)}{n!}
\end{equation*}
[例1] 硬幣 6 個,可擲得多少種不同方形?

(解) 每個硬幣可以出現之面,僅有正與背兩種,故 $n=2$,硬幣 6 個,每擲一次應有六面出現,故 $r=6$,則本題即為自兩種面中,每次取 6 面之重複組合,故擲出不同之方式有 $_2H_6=_{2+601}C_6=_7C_6=7$ 種。

[例2] 四變數之 $n$ 次完全齊次多項式有若干項?

[解] 所求之數等於從四變數允許重複取出 $n$ 個而作之組合數
\begin{equation*}
_4H_n=_{4+n-1}C_n=_{n+3}C_n=\dfrac{(n+3)(n+2)(n+1)}{3!}
\end{equation*}

\textbf{定理二}:$p+q$ 個物件中,有 $p$ 個相同,則每次取 $r(p>r,q>r)$ 個之組合數為
\begin{equation*}
_qC_r+_qC_{r-1}+_qC_{r-2}+\cdots+_qC_1+1
\end{equation*}
\section{組合關係}
(一) $\boldsymbol{_nC_r=_{n-1}C_r+_{n-1}C_{r-1}}$ 

(二) $\boldsymbol{_{m+n}C_r=_mC_r+_mC_{r-1}\times _nC_1+_mC_{r-2}\times _nC_2 + \cdots +_mC_1\times _nC_{r-1}+_nC_r}$

任取 $m+n$ 個物件,可分為兩組:第一組為 $m$ 個物件,第二組為 $n$ 個物件。今在此 $m+n$ 個物件中,選擇 $r$ 個組合,則其組合數可分為下列 $r+1$ 類。即
\begin{enumerate}
  \item[i] 全在第一組中選擇 $r$ 個組合,其組合數為 $_mC_r$
  \item[ii] 在第一組中選擇 $r-1$ 個,更在第二組中選擇 1 個 ,其組合數為 $_mC_{r-1}\times _nC1$
  \item[iii] 在第一組中選擇 $r-2$ 個,更在第二組中選擇 2 個,其組合數為 $_mC_{r-2}\times _nC_2$。
\end{enumerate}

以後各項,依次在第一組中少取一個,在第二組中多取一個,合成 $r$ 個而組合之,直至最後完全在第二組中取 $r$ 個為止。
$\therefore\qquad _{m+n}C_r=_mC_r+_mC_{r-1}\times _nC_1+_mC_{r-2}\times _nC_2+\cdots+_mC_1\times _nC_{r-1}+_nC_r$
\section{雜例}
